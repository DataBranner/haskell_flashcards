%!TEX TS-program = xelatex
%!TEX encoding = UTF-8 Unicode

% Version of 20161029. Derived from file ``Math flashcard template 20111118.tex''
% Packages loaded may be more numerous than needed. Pruning pending.

% Uses a custom command \fl{question}{answer} to generate flashcards. 
% Intended for printing on double-sided paper.

\documentclass[avery5371,grid]{flashcards}
\usepackage{geometry}             % See geometry.pdf to learn the layout options.
\geometry{letterpaper}            % ... or a4paper or a5paper or ... 
\usepackage{mathtools}
\usepackage{graphicx}
\usepackage{amsmath}
\usepackage{mathspec}			% replaces amssymb; do not use unicode-math
\usepackage{fontspec}
\usepackage{mdwtab}				% allows the use of footnotes in tables
\usepackage[normalem]{ulem}		% underlining (with ``normal emphasis'' retained)

\usepackage{datetime}		% allows \currenttime

\defaultfontfeatures{Mapping=tex-text}
\setromanfont[Scale=1,Mapping=tex-text]{TeX Gyre Termes}
\setsansfont[Scale=MatchLowercase,Mapping=tex-text]{TeX Gyre Heros Cn}
\usepackage[noendash,italic]{mathastext}	% propagates document text font to math mode; use option italic to get italic text in math mode

\setlength{\parindent}{0cm}	% no paragraph indent
\setlength{\doublerulesep}{\arrayrulewidth}

\cardfrontstyle{headings}
\cardfrontfoot{(Haskell library functions) \today}
\cardfrontfootstyle[\footnotesize]{right}
\cardbackstyle{empty}

%%%%%%%%%%%%%%%
%%%%%%%%%%%%%%%
%%%%%%%%%%%%%%%
%%%%%%%%%%%%%%%

%normal card: \fl
\newcommand{\fl}[2]{\begin{flashcard}{#1}{\begin{array}{llll}#2\end{array}}\end{flashcard}}

\newcommand\gr[2]{\scalebox{#2}{\includegraphics{#1}}}

\newcommand\doubleplus{+\kern-1.1ex+\kern0.5ex}

%\title{flashcards}
%\author{David Prager Branner}

\DeclareGraphicsRule{.tif}{png}{.png}{`convert #1 `dirname #1`/`basename #1 .tif`.png}

\begin{document}

\fl {product}{
product			&::	&	Num~a~\Rightarrow [a] \rightarrow~a\\
product~[]		&=	&	1\\
product~(n:ns)	&=	&	n * product~ns
}

\fl{length}{
length			&::	&	[a] \rightarrow Int\\
length~[]		&=	&	0\\
length~[~\_:xs]	&=	&	1+length~xs
}

\fl{reverse}{
reverse			&::	&	[a] \rightarrow [a]\\
reverse~[]		&=	&	[]\\
reverse~(x:xs)	&=	&	reverse~xs \doubleplus [x]
}

\fl{insert}{
insert									&::	&	Ord~a \Rightarrow a \rightarrow [a] \rightarrow [a]\\
insert~x~[]								&=	&	[x]\\
insert~x~(y:ys)~|~x\le y				&=	&	x:y:ys\\
~~~~~~~~~~~~~~~~~~~~~~~~~|~otherwise	&=	&	y:insert~x~ys
}

\fl{zip}{
zip					&::	&	[a] \rightarrow [b] \rightarrow [(a, b)]\\
zip~[]~\_			&==	&	[]\\
zip~\_~[]			&==	&	[]\\
zip~(x:xs)~(y:ys)	&==	&	(x, y) : zip~xs~ys
}

\fl{drop}{
drop				&::	&	Int \rightarrow [a] \rightarrow [a]\\
drop~0~xs			&==	&	xs\\
drop~(n+1)~[]		&==	&	[]\\
drop~(n+1)~(~\_:xs)	&==	&	drop~n~xs
}

\fl{map (basic)}{
map					&::	&	(a \rightarrow b) \rightarrow [a] \rightarrow [b]\\
map~f~xs			&==	&	[f~x~|~x \leftarrow xs]
}

\fl{map (recursive)}{
map					&::	&	(a \rightarrow b) \rightarrow [a] \rightarrow [b]\\
map~f~[]			&==	&	[]\\
map~f~xs			&==	&	f~x:map~f~xs
}

\fl{filter}{
filter				&::	&	(a \rightarrow Bool) \rightarrow [a] \rightarrow [a]\\
filter~p~xs			&==	&	[x~|~x \leftarrow xs, p~x]\\
}

\fl{filter (recursive)}{
filter				&::	&	(a \rightarrow Bool) \rightarrow [a] \rightarrow [a]\\
filter~p~[]			&=	&	[]\\
filter~p~(x:xs)~|~p~x			&==	&	x:filter~p~xs\\
~~~~~~~~~~~~~~~~~~~~~~~~|~otherwise	&==	&	filter~p~xs
}

%\fl{qqq}{rrr}

\end{document}